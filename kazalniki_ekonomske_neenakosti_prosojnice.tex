\documentclass[10pt]{beamer}
\usepackage[slovene]{babel}
\usepackage[utf8]{inputenc}
\usepackage[T1]{fontenc}
\usepackage{lmodern}
\usepackage{mathptmx}
\usepackage{helvet}
\usepackage{courier}
\usepackage{hyperref}
\usepackage{tikz}

\usetheme{CambridgeUS}

\begin{document}

\title[Kazalniki ekonomske neenakosti]{Kazalniki ekonomske neenakosti}
\author{Eva Deželak, Žan Jarc, Veronika Sovdat in Ines Šilc}
\institute [FMF]{ Fakulteta za matematiko in fiziko}

\begin{frame}
	\titlepage
\end {frame}

\begin{frame}
Vsebina predavanja:
	\begin{itemize}
		\item Definicije osnovnih pojmov
		\item Tekma med izobraževanjem in tehnologijo
		\item Posamezni kazalniki neenakosti
			\begin{itemize}
				\item Prag revščine
				\item Lorenzova krivulja
				\item Ginijev koeficient
				\item Palmovo razmerje
			\end{itemize}
		\item Ekonomska neenakost doma in po svetu
		\item Ekonomska neenakost in demokracija
	\end{itemize}
\end {frame}

\begin{frame}
\frametitle{Definicije osnovnih pojmov}
\begin{itemize}
\item \textbf{Državni dohodek} je vsota vseh dohodkov, ki so na voljo prebivalcem neke države v danem letu, ne glede na klasifikacijo dohodkov.
$$
\textbf{Državni dohodek} = \textbf{BDP} - \textbf{cena kapitala} + \textbf{neto dobiček iz tujine}
$$
\item \textbf{Državno premoženje} je vsota privatnega in javnega premoženja.
$$
\textbf{Državno premoženje}~=~\textit{domači kapital}~+~\textit{neto tuji kapital}
$$
\item Neenakost v odnosu do dela in kapitala. Plače so ena oblika dohodka iz dela, dohodek iz kapitala pa predstavlja vse prihodke, ki jih ima lahko nek posameznik od kapitala, to so lahko na primer najemnine, dividende, obresti\dots
\end{itemize}
\end{frame}

\begin{frame}
\frametitle{Tekma med izobraževanjem in tehnologijo}
\begin{itemize}
\item Teorija ima dve predpostavki:
\begin{enumerate}
\item Plača delavca je enaka njegovi \textbf{mejni produktivnosti}.
\item Produktivnost delavca je odvisna od njegovih spretnosti in strokovnega znanja ter \textbf{ponudbi} teh spretnosti v dani družbi.
\end{enumerate}


\item Teorija o tekmi med izobraževanjem temelji predvsem na ponudbi in povpraševanju strokovnega znanja v neki državi. 
\item Tehnološki napredek je odvisen od hitrosti inovacij in hitrosti implementacije teh inovacij. Ponavadi dvigne povpraševanje za nova strokovna znanja in ustvari nova delovna mesta.


\item Pomanjkljivosti:
\begin{enumerate}
\item Plača $\neq$ mejna produktivnost
\item Ne razloži gromozanskih razlik med plačami direktorjev in delavci podjetja.
\end{enumerate}
\end{itemize}
\end{frame}


\begin{frame}
\frametitle{Viri}
	\begin{itemize}
		\item
			\label{Pickety}
			T.~Pickety, \emph{Capital in the twenty-first century}, The Belknap Press of 						Harvard University Press, London, 2014.

		\item 
			\label{Razdelitev premoženja}
			\emph{Distribution of wealth}, v: Wikipedia, The Free Encyclopedia, [ogled 							20.~4.~2019], dostopno na \url{https://en.wikipedia.org/w/index.php?								title=Distribution_of_wealth&oldid=854198883}.

		\item 
			\label{Metrike ekonomske neenakosti}
			\emph{Income inequality metrics}, v: Wikipedia, The Free Encyclopedia, [ogled 					20.~4.~2019], dostopno na \url{https://en.wikipedia.org/w/index.php?								title=Income_inequality_metrics&oldid=853906711}.
\end{itemize}
\end {frame}










\end{document}